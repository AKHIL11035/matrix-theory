`%iffalse
\let\negmedspace\undefined
\let\negthickspace\undefined
\documentclass[journal,12pt,twocolumn]{IEEEtran}
\usepackage{cite}
\usepackage{amsmath,amssymb,amsfonts,amsthm}
\usepackage{algorithmic}
\usepackage{graphicx}
\usepackage{textcomp}
\usepackage{xcolor}
\usepackage{txfonts}
\usepackage{listings}
\usepackage{enumitem}
\usepackage{mathtools}
\usepackage{gensymb}
\usepackage{comment}
\usepackage[breaklinks=true]{hyperref}
\usepackage{tkz-euclide} 
\usepackage{listings}
\usepackage{gvv}                                        
%\def\inputGnumericTable{}                                 
\usepackage[latin1]{inputenc}                                
\usepackage{color}                                            
\usepackage{array}                                            
\usepackage{longtable}                                       
\usepackage{calc}                                             
\usepackage{multirow}                                         
\usepackage{hhline}                                           
\usepackage{ifthen}                                           
\usepackage{lscape}
\usepackage{tabularx}
\usepackage{array}
\usepackage{float}


\newtheorem{theorem}{Theorem}[section]
\newtheorem{problem}{Problem}
\newtheorem{proposition}{Proposition}[section]
\newtheorem{lemma}{Lemma}[section]
\newtheorem{corollary}[theorem]{Corollary}
\newtheorem{example}{Example}[section]
\newtheorem{definition}[problem]{Definition}
\newcommand{\BEQA}{\begin{eqnarray}}
\newcommand{\EEQA}{\end{eqnarray}}
\newcommand{\define}{\stackrel{\triangle}{=}}
\theoremstyle{remark}
\newtheorem{rem}{Remark}

% Marks the beginning of the document
\begin{document}
\bibliographystyle{IEEEtran}
\vspace{3cm}

\title{CONIC SECTION}
\author{EE24BTECH11035 - KOTHAPALLI AKHIL}
\maketitle
\newpage
\bigskip

\renewcommand{\thefigure}{\theenumi}
\renewcommand{\thetable}{\theenumi}
\section{SECTION G}
\begin{enumerate}
\item The equation of the locus of the point whose distances from the point P and the line AB are equal, is

\begin{enumerate}
     \item $9x^2+y^2-6xy-54x-62y+241=0$
     \item $x^2+9y^2+6xy-54x-62y-241=0$
     \item $9x^2+9y^2-6xy-54x-62y-241=0$
     \item $x^2+y^2-2xy+27x+31y-120=0$
\end{enumerate}
\section[]{passage 4}
\begin{enumerate}
\item[]Let PQ be a focal chord of the parabola $y^2=4ax$.The tangents to the parabola at P and Q meet at a point lying on the line $y=2x+a$,$a>0$
\item Lenth of the chord PQ is
\hfill(JEE Adv.2013)        
\begin{enumerate}
    \item 7a
    \item 5a
    \item 2a
    \item 3a
\end{enumerate}
\item If chord PQ subtends an angle $\theta$  at the vertex of $y^2=4ax$
\hfill(JEE Adv.2013)
\begin{enumerate}
    \item $\frac{2}{3}\sqrt{7}$
    
    \item $\frac{-2}{3}\sqrt{7}$
    
    \item $\frac{2}{3}\sqrt{5}$
    
    \item $\frac{-2}{3}\sqrt{5}$
\end{enumerate}
\section{passage 5}
\begin{enumerate}
\item[]  Let a,r,s,t be nonzero real numbers. Lets P$(at^2,2as)$,Q,R$(as^2,2as)$ be distinct points on the parabola $y^2=4ax$.suppose that PQ is the focal chord and lines QR and PK are parallel,where K is the point $(2a,0)$
\item The value of r is 
\hfill(JEE Adv.2014)
\begin{enumerate}
    \item $\frac{-1}{t}$\\ 
    \item $\frac{t^2+1}{t}$\\
    \item $\frac{1}{t}$\\
    \item $\frac{t^2-1}{t}$
\end{enumerate}
\item If $st=1$, then the tangent at P and the normal at S to the
parabola meet at a point whose ordinate is 
\begin{enumerate}
    \item $\frac{a(t^2+1)^2}{t^3}$\\
    \item $\frac{a(t^2+1)}{2t^3}$\\
    \item $\frac{1}{t}$\\
    \item $\frac{t^2-1}{t}$
\end{enumerate}
\section{passage 6}
\begin{enumerate}
\item[]  Let ${F_1}$($x_1$,0) and  ${F_2}$($x_2$,0) for $x_1<0$ and $x_2>0$, be the focii of the ellipse $x^2/9 +y^2/8 =1$. Suppose a parabola having vertex at the origin and focus at $F_2$ intersects the ellipse at point M in the first quadrant and the point N in the first quadrant.
\item The orthocentre of the triangle $F_1$MN is

         \hfill(JEE Adv. 2016)
\begin{enumerate}
     \item $(\frac{-9}{10},0)$\\    
     \item $(\frac{2}{3},0)$\\     
     \item $($9/10$,0)$\\ 
    \item $(\frac{2}{3},\sqrt{6})$ 
    \end{enumerate}
\item If the tangents to the ellikpse at M and N meet at R and the normal to the parabola at M meets the X-Axis at Q, the the ratio of area of the triangle MQR to the area of the quadrilateral M$F_1$N$F_2$ is
\hfill(JEE Adv.2016)
\begin{enumerate}
    \item 3:4
    \item 4:5
    \item 5:8
    \item 2:3
\end{enumerate}
\section{ASSERTION AND REASON TYPE QUESTIONS}
\begin{enumerate}
\item STATEMENT-1: The curve $y=\frac{-x^2}{2}+x+1$ is symmetric with respect to the line $x=1$.because
STATEMENT-2: A Parabola is symmetric about its axis.
\hfill(2007-3 marks)
\begin{enumerate}
    \item Statement-1 is True,Statement-2 is True;Statement-2 is a correct explanation for Statement-1
    \item  Statement-1 is True,Statement-2 is True;Statement-2 is NOT a correct explanation for Statement-1
    \item Statement-1 is True,Statement-2 is False
    \item Statement-1 is False,Statement-2 is True
\end{enumerate}
\end{enumerate}
\section{INTEGER VALUE CORRECT TYPE}
\begin{enumerate}
\item The lime $2x+y=1$ is the tangent to the hyperbola $\frac{x^2}{a^2}-\frac{y^2}{b^2}=1$. If this line passes through the point of intersection of the nearest directrix and the X-axis, then the eccentricity of the hyperbola is
\hfill(2010)\\
\item Consider the parabola $y^2=8x$ . Let $\Delta_1$ be the area of the triangle formed by the end poiNts of its latus rectum and the point p$(\frac{1}{2},2)$ on the parabola and $\Delta_2$ be the area of the triangle formed by drawing tangents at P and at the end points of the latus rectum.Then $\frac{\Delta_1}{\Delta_2}$ is 
\hfill(2011)\\
\item Let S be the focus of the parabola $y^2=8x$ and let PQ be the common chord of the circle $x^2+y^2-2x-4y=0$ and the given parabola. The area of the triangle PQS is
\hfill(2012)\\
\item A Vertical line passing through point $(h,0)$ intersects the ellipse   at the points  P and Q . Let the tangents to the ellipse at P and Q meet at the points R.If $\Delta(h)$= area of the triangle PQR, $\Delta_1$= ma
then 
\hfill(JEE Adv.2013)
\begin{enumerate}
    \item g(x) is continuous but not differentiable at a
    \item g(x) is differentiable on R
    \item g(x) is continuous but not differentiable at b
    \item g(x) is continuous and differentiable either(a) or (b) but not both \\
\item If the normal of the parabola $y^2=4x$ drawn at the end points of its latusrectum are the tangents of th circle $(x-3)^2+(y+2)^2=r^2$, then the value of $r^2$ is
\hfill(JEE Adv.2015)\\
\item Let the curve C be the mirror image of the parabola $y^2=4x$ with respect to the line $x+y+4=0$.IfA and B are the points of the intersection of C with the line $y=-5$,then the distance between A and B is
\hfill(JEE Adv.2015)\\
\item Suppose that the focii of the ellipse $\frac{x^2}{9}+\frac{y^2}{5}=1$ are ($f_1$,0) and ($f_2$,0) where $f_1>0$ and $f_1<0$.Let $P_1$ and $P_2$ be two parabolas with a common vertex at $(0,0)$ and with foci at ($f_1$,0) and (2$f_2$,0),respectively. Let $T_1$ be a tangent to $P_1$ which passes through (2$f_2$,0) and $T_2$ be a tangent to $P_2$ which passes through ($f_1$,0).If $m_1$ is the slope of $T_1$ and $m_2$ is the slope of $T_2$,then the value of
\hfill(JEE Adv. 2015)
\end{enumerate}


\end{document}


    
    
     

`%iffalse
\let\negmedspace\undefined
\let\negthickspace\undefined
\documentclass[journal,12pt,twocolumn]{IEEEtran}
\usepackage{cite}
\usepackage{amsmath,amssymb,amsfonts,amsthm}
\usepackage{algorithmic}
\usepackage{graphicx}
\usepackage{textcomp}
\usepackage{xcolor}
\usepackage{txfonts}
\usepackage{listings}
\usepackage{enumitem}
\usepackage{mathtools}
\usepackage{gensymb}
\usepackage{comment}
\usepackage[breaklinks=true]{hyperref}
\usepackage{tkz-euclide} 
\usepackage{listings}
\usepackage{gvv}                                        
%\def\inputGnumericTable{}                                 
\usepackage[latin1]{inputenc}                                
\usepackage{color}                                            
\usepackage{array}                                            
\usepackage{longtable}                                       
\usepackage{calc}                                             
\usepackage{multirow}                                         
\usepackage{hhline}                                           
\usepackage{ifthen}                                           
\usepackage{lscape}
\usepackage{tabularx}
\usepackage{array}
\usepackage{float}


\newtheorem{theorem}{Theorem}[section]
\newtheorem{problem}{Problem}
\newtheorem{proposition}{Proposition}[section]
\newtheorem{lemma}{Lemma}[section]
\newtheorem{corollary}[theorem]{Corollary}
\newtheorem{example}{Example}[section]
\newtheorem{definition}[problem]{Definition}
\newcommand{\BEQA}{\begin{eqnarray}}
\newcommand{\EEQA}{\end{eqnarray}}
\newcommand{\define}{\stackrel{\triangle}{=}}
\theoremstyle{remark}
\newtheorem{rem}{Remark}

% Marks the beginning of the document
\begin{document}
\bibliographystyle{IEEEtran}
\vspace{3cm}

\title{CONIC SECTION}
\author{EE24BTECH11035 - KOTHAPALLI AKHIL}
\maketitle
\newpage
\bigskip

\renewcommand{\thefigure}{\theenumi}
\renewcommand{\thetable}{\theenumi}
\section{SECTION G}
\begin{enumerate}
\item[8.] The equation of the locus of the point whose distances from the point P and the line AB are equal, is

\begin{enumerate}
     \item $9x^2+y^2-6xy-54x-62y+241=0$
     \item $x^2+9y^2+6xy-54x-62y-241=0$
     \item $9x^2+9y^2-6xy-54x-62y-241=0$
     \item $x^2+y^2-2xy+27x+31y-120=0$
\end{enumerate}
\section[]{passage 4}     
\item[]Let PQ be a focal chord of the parabola $y^2=4ax$.The tangents to the parabola at P and Q meet at a point lying on the line $y=2x+a$,$a>0$
\item[9.] Lenth of the chord PQ is
\hfill(JEE Adv.2013)        
\begin{enumerate}
    \item 7a
    \item 5a
    \item 2a
    \item 3a
\end{enumerate}
\item[10.] If chord PQ subtends an angle $\theta$  at the vertex of $y^2=4ax$
\hfill(JEE Adv.2013)
\begin{enumerate}
    \item $\frac{2}{3}\sqrt{7}$
    
    \item $\frac{-2}{3}\sqrt{7}$
    
    \item $\frac{2}{3}\sqrt{5}$
    
    \item $\frac{-2}{3}\sqrt{5}$
\end{enumerate}
\section{passage 5}
\item[]  Let a,r,s,t be nonzero real numbers. Lets P$(at^2,2as)$,Q,R$(as^2,2as)$ be distinct points on the parabola $y^2=4ax$.suppose that PQ is the focal chord and lines QR and PK are parallel,where K is the point $(2a,0)$
\item[11.] The value of r is 
\hfill(JEE Adv.2014)
\begin{enumerate}
    \item $\frac{-1}{t}$\\ 
    \item $\frac{t^2+1}{t}$\\
    \item $\frac{1}{t}$\\
    \item $\frac{t^2-1}{t}$
\end{enumerate}
\item[12.] If $st=1$, then the tangent at P and the normal at S to the
parabola meet at a point whose ordinate is 
\begin{enumerate}
    \item $\frac{a(t^2+1)^2}{t^3}$\\
    \item $\frac{a(t^2+1)}{2t^3}$\\
    \item $\frac{1}{t}$\\
    \item $\frac{t^2-1}{t}$
    \end{enumerate}
\section{passage 6}
\item[]  Let ${F_1}$($x_1$,0) and  ${F_2}$($x_2$,0) for $x_1<0$ and $x_2>0$, be the focii of the ellipse $x^2/9 +y^2/8 =1$. Suppose a parabola having vertex at the origin and focus at $F_2$ intersects the ellipse at point M in the first quadrant and the point N in the first quadrant.
\item[13.] The orthocentre of the triangle $F_1$MN is

         \hfill(JEE Adv. 2016)
\begin{enumerate}
     \item $(\frac{-9}{10},0)$\\    
     \item $(\frac{2}{3},0)$\\     
     \item $($9/10$,0)$\\ 
    \item $(\frac{2}{3},\sqrt{6})$ 
    \end{enumerate}
\item[14.] If the tangents to the ellikpse at M and N meet at R and the normal to the parabola at M meets the X-Axis at Q, the the ratio of area of the triangle MQR to the area of the quadrilateral M$F_1$N$F_2$ is
\hfill(JEE Adv.2016)
\begin{enumerate}
    \item 3:4
    \item 4:5
    \item 5:8
    \item 2:3
\end{enumerate}
\section{ASSERTION AND REASON TYPE QUESTIONS}
\item[1.] STATEMENT-1: The curve $y=\frac{-x^2}{2}+x+1$ is symmetric with respect to the line $x=1$.because
STATEMENT-2: A Parabola is symmetric about its axis.
\hfill(2007-3 marks)
\begin{enumerate}
    \item Statement-1 is True,Statement-2 is True;Statement-2 is a correct explanation for Statement-1\item  Statement-1 is True,Statement-2 is True;Statement-2 is NOT a correct explanation for Statement-1\item Statement-1 is True,Statement-2 is False\item Statement-1 is False,Statement-2 is True
\end{enumerate}
\section{INTEGER VALUE CORRECT TYPE}
\item[1.] The lime $2x+y=1$ is the tangent to the hyperbola $\frac{x^2}{a^2}-\frac{y^2}{b^2}=1$. If this line passes through the point of intersection of the nearest directrix and the X-axis, then the eccentricity of the hyperbola is
\hfill(2010)\\
\item[2.] Consider the parabola $y^2=8x$ . Let $\Delta_1$ be the area of the triangle formed by the end poiNts of its latus rectum and the point p$(\frac{1}{2},2)$ on the parabola and $\Delta_2$ be the area of the triangle formed by drawing tangents at P and at the end points of the latus rectum.Then $\frac{\Delta_1}{\Delta_2}$ is 
\hfill(2011)\\
\item[3.] Let S be the focus of the parabola $y^2=8x$ and let PQ be the common chord of the circle $x^2+y^2-2x-4y=0$ and the given parabola. The area of the triangle PQS is
\hfill(2012)\\
\item[4.]A Vertical line passing through point $(h,0)$ intersects the ellipse   at the points  P and Q . Let the tangents to the ellipse at P and Q meet at the points R.If $\Delta(h)$= area of the triangle PQR, $\Delta_1$= ma
then 
\hfill(JEE Adv.2013)
\begin{enumerate}
    \item g(x) is continuous but not differentiable at a
    \item g(x) is differentiable on R
    \item g(x) is continuous but not differentiable at b
    \item g(x) is continuous and differentiable either(a) or (b) but not both \\
\item[5.] If the normal of the parabola $y^2=4x$ drawn at the end points of its latusrectum are the tangents of th circle $(x-3)^2+(y+2)^2=r^2$, then the value of $r^2$ is
\hfill(JEE Adv.2015)\\
\item[6.] Let the curve C be the mirror image of the parabola $y^2=4x$ with respect to the line $x+y+4=0$.IfA and B are the points of the intersection of C with the line $y=-5$,then the distance between A and B is
\hfill(JEE Adv.2015)\\
\item[7.]Suppose that the focii of the ellipse $\frac{x^2}{9}+\frac{y^2}{5}=1$ are ($f_1$,0) and ($f_2$,0) where $f_1>0$ and $f_1<0$.Let $P_1$ and $P_2$ be two parabolas with a common vertex at $(0,0)$ and with foci at ($f_1$,0) and (2$f_2$,0),respectively. Let $T_1$ be a tangent to $P_1$ which passes through (2$f_2$,0) and $T_2$ be a tangent to $P_2$ which passes through ($f_1$,0).If $m_1$ is the slope of $T_1$ and $m_2$ is the slope of $T_2$,then the value of
\hfill(JEE Adv. 2015)
\end{enumerate}


\end{document}


    
    
     

`%iffalse
\let\negmedspace\undefined
\let\negthickspace\undefined
\documentclass[journal,12pt,twocolumn]{IEEEtran}
\usepackage{cite}
\usepackage{amsmath,amssymb,amsfonts,amsthm}
\usepackage{algorithmic}
\usepackage{graphicx}
\usepackage{textcomp}
\usepackage{xcolor}
\usepackage{txfonts}
\usepackage{listings}
\usepackage{enumitem}
\usepackage{mathtools}
\usepackage{gensymb}
\usepackage{comment}
\usepackage[breaklinks=true]{hyperref}
\usepackage{tkz-euclide} 
\usepackage{listings}
\usepackage{gvv}                                        
%\def\inputGnumericTable{}                                 
\usepackage[latin1]{inputenc}                                
\usepackage{color}                                            
\usepackage{array}                                            
\usepackage{longtable}                                       
\usepackage{calc}                                             
\usepackage{multirow}                                         
\usepackage{hhline}                                           
\usepackage{ifthen}                                           
\usepackage{lscape}
\usepackage{tabularx}
\usepackage{array}
\usepackage{float}


\newtheorem{theorem}{Theorem}[section]
\newtheorem{problem}{Problem}
\newtheorem{proposition}{Proposition}[section]
\newtheorem{lemma}{Lemma}[section]
\newtheorem{corollary}[theorem]{Corollary}
\newtheorem{example}{Example}[section]
\newtheorem{definition}[problem]{Definition}
\newcommand{\BEQA}{\begin{eqnarray}}
\newcommand{\EEQA}{\end{eqnarray}}
\newcommand{\define}{\stackrel{\triangle}{=}}
\theoremstyle{remark}
\newtheorem{rem}{Remark}

% Marks the beginning of the document
\begin{document}
\bibliographystyle{IEEEtran}
\vspace{3cm}

\title{CONIC SECTION}
\author{EE24BTECH11035 - KOTHAPALLI AKHIL}
\maketitle
\newpage
\bigskip

\renewcommand{\thefigure}{\theenumi}
\renewcommand{\thetable}{\theenumi}
\section{SECTION G}
\begin{enumerate}
\item[8.] The equation of the locus of the point whose distances from the point P and the line AB are equal, is

\begin{enumerate}
     \item $9x^2+y^2-6xy-54x-62y+241=0$
     \item $x^2+9y^2+6xy-54x-62y-241=0$
     \item $9x^2+9y^2-6xy-54x-62y-241=0$
     \item $x^2+y^2-2xy+27x+31y-120=0$
\end{enumerate}
\section[]{passage 4}     
\item[]Let PQ be a focal chord of the parabola $y^2=4ax$.The tangents to the parabola at P and Q meet at a point lying on the line $y=2x+a$,$a>0$
\item[9.] Lenth of the chord PQ is
\hfill(JEE Adv.2013)        
\begin{enumerate}
    \item 7a
    \item 5a
    \item 2a
    \item 3a
\end{enumerate}
\item[10.] If chord PQ subtends an angle $\theta$  at the vertex of $y^2=4ax$
\hfill(JEE Adv.2013)
\begin{enumerate}
    \item $\frac{2}{3}\sqrt{7}$
    
    \item $\frac{-2}{3}\sqrt{7}$
    
    \item $\frac{2}{3}\sqrt{5}$
    
    \item $\frac{-2}{3}\sqrt{5}$
\end{enumerate}
\section{passage 5}
\item[]  Let a,r,s,t be nonzero real numbers. Lets P$(at^2,2as)$,Q,R$(as^2,2as)$ be distinct points on the parabola $y^2=4ax$.suppose that PQ is the focal chord and lines QR and PK are parallel,where K is the point $(2a,0)$
\item[11.] The value of r is 
\hfill(JEE Adv.2014)
\begin{enumerate}
    \item $\frac{-1}{t}$\\ 
    \item $\frac{t^2+1}{t}$\\
    \item $\frac{1}{t}$\\
    \item $\frac{t^2-1}{t}$
\end{enumerate}
\item[12.] If $st=1$, then the tangent at P and the normal at S to the
parabola meet at a point whose ordinate is 
\begin{enumerate}
    \item $\frac{a(t^2+1)^2}{t^3}$\\
    \item $\frac{a(t^2+1)}{2t^3}$\\
    \item $\frac{1}{t}$\\
    \item $\frac{t^2-1}{t}$
    \end{enumerate}
\section{passage 6}
\item[]  Let ${F_1}$($x_1$,0) and  ${F_2}$($x_2$,0) for $x_1<0$ and $x_2>0$, be the focii of the ellipse $x^2/9 +y^2/8 =1$. Suppose a parabola having vertex at the origin and focus at $F_2$ intersects the ellipse at point M in the first quadrant and the point N in the first quadrant.
\item[13.] The orthocentre of the triangle $F_1$MN is

         \hfill(JEE Adv. 2016)
\begin{enumerate}
     \item $(\frac{-9}{10},0)$\\    
     \item $(\frac{2}{3},0)$\\     
     \item $($9/10$,0)$\\ 
    \item $(\frac{2}{3},\sqrt{6})$ 
    \end{enumerate}
\item[14.] If the tangents to the ellikpse at M and N meet at R and the normal to the parabola at M meets the X-Axis at Q, the the ratio of area of the triangle MQR to the area of the quadrilateral M$F_1$N$F_2$ is
\hfill(JEE Adv.2016)
\begin{enumerate}
    \item 3:4
    \item 4:5
    \item 5:8
    \item 2:3
\end{enumerate}
\section{ASSERTION AND REASON TYPE QUESTIONS}
\item[1.] STATEMENT-1: The curve $y=\frac{-x^2}{2}+x+1$ is symmetric with respect to the line $x=1$.because
STATEMENT-2: A Parabola is symmetric about its axis.
\hfill(2007-3 marks)
\begin{enumerate}
    \item Statement-1 is True,Statement-2 is True;Statement-2 is a correct explanation for Statement-1\item  Statement-1 is True,Statement-2 is True;Statement-2 is NOT a correct explanation for Statement-1\item Statement-1 is True,Statement-2 is False\item Statement-1 is False,Statement-2 is True
\end{enumerate}
\section{INTEGER VALUE CORRECT TYPE}
\item[1.] The lime $2x+y=1$ is the tangent to the hyperbola $\frac{x^2}{a^2}-\frac{y^2}{b^2}=1$. If this line passes through the point of intersection of the nearest directrix and the X-axis, then the eccentricity of the hyperbola is
\hfill(2010)\\
\item[2.] Consider the parabola $y^2=8x$ . Let $\Delta_1$ be the area of the triangle formed by the end poiNts of its latus rectum and the point p$(\frac{1}{2},2)$ on the parabola and $\Delta_2$ be the area of the triangle formed by drawing tangents at P and at the end points of the latus rectum.Then $\frac{\Delta_1}{\Delta_2}$ is 
\hfill(2011)\\
\item[3.] Let S be the focus of the parabola $y^2=8x$ and let PQ be the common chord of the circle $x^2+y^2-2x-4y=0$ and the given parabola. The area of the triangle PQS is
\hfill(2012)\\
\item[4.]A Vertical line passing through point $(h,0)$ intersects the ellipse   at the points  P and Q . Let the tangents to the ellipse at P and Q meet at the points R.If $\Delta(h)$= area of the triangle PQR, $\Delta_1$= ma
then 
\hfill(JEE Adv.2013)
\begin{enumerate}
    \item g(x) is continuous but not differentiable at a
    \item g(x) is differentiable on R
    \item g(x) is continuous but not differentiable at b
    \item g(x) is continuous and differentiable either(a) or (b) but not both \\
\item[5.] If the normal of the parabola $y^2=4x$ drawn at the end points of its latusrectum are the tangents of th circle $(x-3)^2+(y+2)^2=r^2$, then the value of $r^2$ is
\hfill(JEE Adv.2015)\\
\item[6.] Let the curve C be the mirror image of the parabola $y^2=4x$ with respect to the line $x+y+4=0$.IfA and B are the points of the intersection of C with the line $y=-5$,then the distance between A and B is
\hfill(JEE Adv.2015)\\
\item[7.]Suppose that the focii of the ellipse $\frac{x^2}{9}+\frac{y^2}{5}=1$ are ($f_1$,0) and ($f_2$,0) where $f_1>0$ and $f_1<0$.Let $P_1$ and $P_2$ be two parabolas with a common vertex at $(0,0)$ and with foci at ($f_1$,0) and (2$f_2$,0),respectively. Let $T_1$ be a tangent to $P_1$ which passes through (2$f_2$,0) and $T_2$ be a tangent to $P_2$ which passes through ($f_1$,0).If $m_1$ is the slope of $T_1$ and $m_2$ is the slope of $T_2$,then the value of
\hfill(JEE Adv. 2015)
\end{enumerate}


\end{document}


    
    
     

`%iffalse
\let\negmedspace\undefined
\let\negthickspace\undefined
\documentclass[journal,12pt,twocolumn]{IEEEtran}
\usepackage{cite}
\usepackage{amsmath,amssymb,amsfonts,amsthm}
\usepackage{algorithmic}
\usepackage{graphicx}
\usepackage{textcomp}
\usepackage{xcolor}
\usepackage{txfonts}
\usepackage{listings}
\usepackage{enumitem}
\usepackage{mathtools}
\usepackage{gensymb}
\usepackage{comment}
\usepackage[breaklinks=true]{hyperref}
\usepackage{tkz-euclide} 
\usepackage{listings}
\usepackage{gvv}                                        
%\def\inputGnumericTable{}                                 
\usepackage[latin1]{inputenc}                                
\usepackage{color}                                            
\usepackage{array}                                            
\usepackage{longtable}                                       
\usepackage{calc}                                             
\usepackage{multirow}                                         
\usepackage{hhline}                                           
\usepackage{ifthen}                                           
\usepackage{lscape}
\usepackage{tabularx}
\usepackage{array}
\usepackage{float}


\newtheorem{theorem}{Theorem}[section]
\newtheorem{problem}{Problem}
\newtheorem{proposition}{Proposition}[section]
\newtheorem{lemma}{Lemma}[section]
\newtheorem{corollary}[theorem]{Corollary}
\newtheorem{example}{Example}[section]
\newtheorem{definition}[problem]{Definition}
\newcommand{\BEQA}{\begin{eqnarray}}
\newcommand{\EEQA}{\end{eqnarray}}
\newcommand{\define}{\stackrel{\triangle}{=}}
\theoremstyle{remark}
\newtheorem{rem}{Remark}

% Marks the beginning of the document
\begin{document}
\bibliographystyle{IEEEtran}
\vspace{3cm}

\title{CONIC SECTION}
\author{EE24BTECH11035 - KOTHAPALLI AKHIL}
\maketitle
\newpage
\bigskip

\renewcommand{\thefigure}{\theenumi}
\renewcommand{\thetable}{\theenumi}
\section{SECTION G}
\begin{enumerate}
\item[8.] The equation of the locus of the point whose distances from the point P and the line AB are equal, is

\begin{enumerate}
     \item $9x^2+y^2-6xy-54x-62y+241=0$
     \item $x^2+9y^2+6xy-54x-62y-241=0$
     \item $9x^2+9y^2-6xy-54x-62y-241=0$
     \item $x^2+y^2-2xy+27x+31y-120=0$
\end{enumerate}
\section[]{passage 4}     
\item[]Let PQ be a focal chord of the parabola $y^2=4ax$.The tangents to the parabola at P and Q meet at a point lying on the line $y=2x+a$,$a>0$
\item[9.] Lenth of the chord PQ is
\hfill(JEE Adv.2013)        
\begin{enumerate}
    \item 7a
    \item 5a
    \item 2a
    \item 3a
\end{enumerate}
\item[10.] If chord PQ subtends an angle $\theta$  at the vertex of $y^2=4ax$
\hfill(JEE Adv.2013)
\begin{enumerate}
    \item $\frac{2}{3}\sqrt{7}$
    
    \item $\frac{-2}{3}\sqrt{7}$
    
    \item $\frac{2}{3}\sqrt{5}$
    
    \item $\frac{-2}{3}\sqrt{5}$
\end{enumerate}
\section{passage 5}
\item[]  Let a,r,s,t be nonzero real numbers. Lets P$(at^2,2as)$,Q,R$(as^2,2as)$ be distinct points on the parabola $y^2=4ax$.suppose that PQ is the focal chord and lines QR and PK are parallel,where K is the point $(2a,0)$
\item[11.] The value of r is 
\hfill(JEE Adv.2014)
\begin{enumerate}
    \item $\frac{-1}{t}$\\ 
    \item $\frac{t^2+1}{t}$\\
    \item $\frac{1}{t}$\\
    \item $\frac{t^2-1}{t}$
\end{enumerate}
\item[12.] If $st=1$, then the tangent at P and the normal at S to the
parabola meet at a point whose ordinate is 
\begin{enumerate}
    \item $\frac{a(t^2+1)^2}{t^3}$\\
    \item $\frac{a(t^2+1)}{2t^3}$\\
    \item $\frac{1}{t}$\\
    \item $\frac{t^2-1}{t}$
    \end{enumerate}
\section{passage 6}
\item[]  Let ${F_1}$($x_1$,0) and  ${F_2}$($x_2$,0) for $x_1<0$ and $x_2>0$, be the focii of the ellipse $x^2/9 +y^2/8 =1$. Suppose a parabola having vertex at the origin and focus at $F_2$ intersects the ellipse at point M in the first quadrant and the point N in the first quadrant.
\item[13.] The orthocentre of the triangle $F_1$MN is

         \hfill(JEE Adv. 2016)
\begin{enumerate}
     \item $(\frac{-9}{10},0)$\\    
     \item $(\frac{2}{3},0)$\\     
     \item $($9/10$,0)$\\ 
    \item $(\frac{2}{3},\sqrt{6})$ 
    \end{enumerate}
\item[14.] If the tangents to the ellikpse at M and N meet at R and the normal to the parabola at M meets the X-Axis at Q, the the ratio of area of the triangle MQR to the area of the quadrilateral M$F_1$N$F_2$ is
\hfill(JEE Adv.2016)
\begin{enumerate}
    \item 3:4
    \item 4:5
    \item 5:8
    \item 2:3
\end{enumerate}
\section{ASSERTION AND REASON TYPE QUESTIONS}
\item[1.] STATEMENT-1: The curve $y=\frac{-x^2}{2}+x+1$ is symmetric with respect to the line $x=1$.because
STATEMENT-2: A Parabola is symmetric about its axis.
\hfill(2007-3 marks)
\begin{enumerate}
    \item Statement-1 is True,Statement-2 is True;Statement-2 is a correct explanation for Statement-1\item  Statement-1 is True,Statement-2 is True;Statement-2 is NOT a correct explanation for Statement-1\item Statement-1 is True,Statement-2 is False\item Statement-1 is False,Statement-2 is True
\end{enumerate}
\section{INTEGER VALUE CORRECT TYPE}
\item[1.] The lime $2x+y=1$ is the tangent to the hyperbola $\frac{x^2}{a^2}-\frac{y^2}{b^2}=1$. If this line passes through the point of intersection of the nearest directrix and the X-axis, then the eccentricity of the hyperbola is
\hfill(2010)\\
\item[2.] Consider the parabola $y^2=8x$ . Let $\Delta_1$ be the area of the triangle formed by the end poiNts of its latus rectum and the point p$(\frac{1}{2},2)$ on the parabola and $\Delta_2$ be the area of the triangle formed by drawing tangents at P and at the end points of the latus rectum.Then $\frac{\Delta_1}{\Delta_2}$ is 
\hfill(2011)\\
\item[3.] Let S be the focus of the parabola $y^2=8x$ and let PQ be the common chord of the circle $x^2+y^2-2x-4y=0$ and the given parabola. The area of the triangle PQS is
\hfill(2012)\\
\item[4.]A Vertical line passing through point $(h,0)$ intersects the ellipse   at the points  P and Q . Let the tangents to the ellipse at P and Q meet at the points R.If $\Delta(h)$= area of the triangle PQR, $\Delta_1$= ma
then 
\hfill(JEE Adv.2013)
\begin{enumerate}
    \item g(x) is continuous but not differentiable at a
    \item g(x) is differentiable on R
    \item g(x) is continuous but not differentiable at b
    \item g(x) is continuous and differentiable either(a) or (b) but not both \\
\item[5.] If the normal of the parabola $y^2=4x$ drawn at the end points of its latusrectum are the tangents of th circle $(x-3)^2+(y+2)^2=r^2$, then the value of $r^2$ is
\hfill(JEE Adv.2015)\\
\item[6.] Let the curve C be the mirror image of the parabola $y^2=4x$ with respect to the line $x+y+4=0$.IfA and B are the points of the intersection of C with the line $y=-5$,then the distance between A and B is
\hfill(JEE Adv.2015)\\
\item[7.]Suppose that the focii of the ellipse $\frac{x^2}{9}+\frac{y^2}{5}=1$ are ($f_1$,0) and ($f_2$,0) where $f_1>0$ and $f_1<0$.Let $P_1$ and $P_2$ be two parabolas with a common vertex at $(0,0)$ and with foci at ($f_1$,0) and (2$f_2$,0),respectively. Let $T_1$ be a tangent to $P_1$ which passes through (2$f_2$,0) and $T_2$ be a tangent to $P_2$ which passes through ($f_1$,0).If $m_1$ is the slope of $T_1$ and $m_2$ is the slope of $T_2$,then the value of
\hfill(JEE Adv. 2015)
\end{enumerate}


\end{document}


    
    
     

`%iffalse
\let\negmedspace\undefined
\let\negthickspace\undefined
\documentclass[journal,12pt,twocolumn]{IEEEtran}
\usepackage{cite}
\usepackage{amsmath,amssymb,amsfonts,amsthm}
\usepackage{algorithmic}
\usepackage{graphicx}
\usepackage{textcomp}
\usepackage{xcolor}
\usepackage{txfonts}
\usepackage{listings}
\usepackage{enumitem}
\usepackage{mathtools}
\usepackage{gensymb}
\usepackage{comment}
\usepackage[breaklinks=true]{hyperref}
\usepackage{tkz-euclide} 
\usepackage{listings}
\usepackage{gvv}                                        
%\def\inputGnumericTable{}                                 
\usepackage[latin1]{inputenc}                                
\usepackage{color}                                            
\usepackage{array}                                            
\usepackage{longtable}                                       
\usepackage{calc}                                             
\usepackage{multirow}                                         
\usepackage{hhline}                                           
\usepackage{ifthen}                                           
\usepackage{lscape}
\usepackage{tabularx}
\usepackage{array}
\usepackage{float}


\newtheorem{theorem}{Theorem}[section]
\newtheorem{problem}{Problem}
\newtheorem{proposition}{Proposition}[section]
\newtheorem{lemma}{Lemma}[section]
\newtheorem{corollary}[theorem]{Corollary}
\newtheorem{example}{Example}[section]
\newtheorem{definition}[problem]{Definition}
\newcommand{\BEQA}{\begin{eqnarray}}
\newcommand{\EEQA}{\end{eqnarray}}
\newcommand{\define}{\stackrel{\triangle}{=}}
\theoremstyle{remark}
\newtheorem{rem}{Remark}

% Marks the beginning of the document
\begin{document}
\bibliographystyle{IEEEtran}
\vspace{3cm}

\title{CONIC SECTION}
\author{EE24BTECH11035 - KOTHAPALLI AKHIL}
\maketitle
\newpage
\bigskip

\renewcommand{\thefigure}{\theenumi}
\renewcommand{\thetable}{\theenumi}
\section{SECTION G}
\begin{enumerate}
\item[8.] The equation of the locus of the point whose distances from the point P and the line AB are equal, is

\begin{enumerate}
     \item $9x^2+y^2-6xy-54x-62y+241=0$
     \item $x^2+9y^2+6xy-54x-62y-241=0$
     \item $9x^2+9y^2-6xy-54x-62y-241=0$
     \item $x^2+y^2-2xy+27x+31y-120=0$
\end{enumerate}
\section[]{passage 4}     
\item[]Let PQ be a focal chord of the parabola $y^2=4ax$.The tangents to the parabola at P and Q meet at a point lying on the line $y=2x+a$,$a>0$
\item[9.] Lenth of the chord PQ is
\hfill(JEE Adv.2013)        
\begin{enumerate}
    \item 7a
    \item 5a
    \item 2a
    \item 3a
\end{enumerate}
\item[10.] If chord PQ subtends an angle $\theta$  at the vertex of $y^2=4ax$
\hfill(JEE Adv.2013)
\begin{enumerate}
    \item $\frac{2}{3}\sqrt{7}$
    
    \item $\frac{-2}{3}\sqrt{7}$
    
    \item $\frac{2}{3}\sqrt{5}$
    
    \item $\frac{-2}{3}\sqrt{5}$
\end{enumerate}
\section{passage 5}
\item[]  Let a,r,s,t be nonzero real numbers. Lets P$(at^2,2as)$,Q,R$(as^2,2as)$ be distinct points on the parabola $y^2=4ax$.suppose that PQ is the focal chord and lines QR and PK are parallel,where K is the point $(2a,0)$
\item[11.] The value of r is 
\hfill(JEE Adv.2014)
\begin{enumerate}
    \item $\frac{-1}{t}$\\ 
    \item $\frac{t^2+1}{t}$\\
    \item $\frac{1}{t}$\\
    \item $\frac{t^2-1}{t}$
\end{enumerate}
\item[12.] If $st=1$, then the tangent at P and the normal at S to the
parabola meet at a point whose ordinate is 
\begin{enumerate}
    \item $\frac{a(t^2+1)^2}{t^3}$\\
    \item $\frac{a(t^2+1)}{2t^3}$\\
    \item $\frac{1}{t}$\\
    \item $\frac{t^2-1}{t}$
    \end{enumerate}
\section{passage 6}
\item[]  Let ${F_1}$($x_1$,0) and  ${F_2}$($x_2$,0) for $x_1<0$ and $x_2>0$, be the focii of the ellipse $x^2/9 +y^2/8 =1$. Suppose a parabola having vertex at the origin and focus at $F_2$ intersects the ellipse at point M in the first quadrant and the point N in the first quadrant.
\item[13.] The orthocentre of the triangle $F_1$MN is

         \hfill(JEE Adv. 2016)
\begin{enumerate}
     \item $(\frac{-9}{10},0)$\\    
     \item $(\frac{2}{3},0)$\\     
     \item $($9/10$,0)$\\ 
    \item $(\frac{2}{3},\sqrt{6})$ 
    \end{enumerate}
\item[14.] If the tangents to the ellikpse at M and N meet at R and the normal to the parabola at M meets the X-Axis at Q, the the ratio of area of the triangle MQR to the area of the quadrilateral M$F_1$N$F_2$ is
\hfill(JEE Adv.2016)
\begin{enumerate}
    \item 3:4
    \item 4:5
    \item 5:8
    \item 2:3
\end{enumerate}
\section{ASSERTION AND REASON TYPE QUESTIONS}
\item[1.] STATEMENT-1: The curve $y=\frac{-x^2}{2}+x+1$ is symmetric with respect to the line $x=1$.because
STATEMENT-2: A Parabola is symmetric about its axis.
\hfill(2007-3 marks)
\begin{enumerate}
    \item Statement-1 is True,Statement-2 is True;Statement-2 is a correct explanation for Statement-1\item  Statement-1 is True,Statement-2 is True;Statement-2 is NOT a correct explanation for Statement-1\item Statement-1 is True,Statement-2 is False\item Statement-1 is False,Statement-2 is True
\end{enumerate}
\section{INTEGER VALUE CORRECT TYPE}
\item[1.] The lime $2x+y=1$ is the tangent to the hyperbola $\frac{x^2}{a^2}-\frac{y^2}{b^2}=1$. If this line passes through the point of intersection of the nearest directrix and the X-axis, then the eccentricity of the hyperbola is
\hfill(2010)\\
\item[2.] Consider the parabola $y^2=8x$ . Let $\Delta_1$ be the area of the triangle formed by the end poiNts of its latus rectum and the point p$(\frac{1}{2},2)$ on the parabola and $\Delta_2$ be the area of the triangle formed by drawing tangents at P and at the end points of the latus rectum.Then $\frac{\Delta_1}{\Delta_2}$ is 
\hfill(2011)\\
\item[3.] Let S be the focus of the parabola $y^2=8x$ and let PQ be the common chord of the circle $x^2+y^2-2x-4y=0$ and the given parabola. The area of the triangle PQS is
\hfill(2012)\\
\item[4.]A Vertical line passing through point $(h,0)$ intersects the ellipse   at the points  P and Q . Let the tangents to the ellipse at P and Q meet at the points R.If $\Delta(h)$= area of the triangle PQR, $\Delta_1$= ma
then 
\hfill(JEE Adv.2013)
\begin{enumerate}
    \item g(x) is continuous but not differentiable at a
    \item g(x) is differentiable on R
    \item g(x) is continuous but not differentiable at b
    \item g(x) is continuous and differentiable either(a) or (b) but not both \\
\item[5.] If the normal of the parabola $y^2=4x$ drawn at the end points of its latusrectum are the tangents of th circle $(x-3)^2+(y+2)^2=r^2$, then the value of $r^2$ is
\hfill(JEE Adv.2015)\\
\item[6.] Let the curve C be the mirror image of the parabola $y^2=4x$ with respect to the line $x+y+4=0$.IfA and B are the points of the intersection of C with the line $y=-5$,then the distance between A and B is
\hfill(JEE Adv.2015)\\
\item[7.]Suppose that the focii of the ellipse $\frac{x^2}{9}+\frac{y^2}{5}=1$ are ($f_1$,0) and ($f_2$,0) where $f_1>0$ and $f_1<0$.Let $P_1$ and $P_2$ be two parabolas with a common vertex at $(0,0)$ and with foci at ($f_1$,0) and (2$f_2$,0),respectively. Let $T_1$ be a tangent to $P_1$ which passes through (2$f_2$,0) and $T_2$ be a tangent to $P_2$ which passes through ($f_1$,0).If $m_1$ is the slope of $T_1$ and $m_2$ is the slope of $T_2$,then the value of
\hfill(JEE Adv. 2015)
\end{enumerate}


\end{document}


    
    
     


